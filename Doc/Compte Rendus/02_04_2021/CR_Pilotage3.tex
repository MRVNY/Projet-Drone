\documentclass{article}
\usepackage[utf8]{inputenc}

\title{Compte rendu Pilotage, réunion du 2 Avril}
\author{Johan Rossi, Qingyuan Yao}

\date{April 3 2021}

\begin{document}

\maketitle

\section{Architecture:}
\begin{itemize}
    \item Gérer le l'interface C/C++ du projet. Créer un main en c++, définir ce qu'il faut wrapper , explorer les différente méthodes ...
    \item Créer un commun.h, variables de communication entre les parties...
    \item Déménager les commandes de contrôle dans une fonction callback séparée
\end{itemize}

\section{Création de bouchons}
\begin{itemize}
    \item Une fois l'architecture fixée, on démarre les protos (Méthodes de proto déjà définies)
\end{itemize}

\section{Implémenter un watchdog}
\begin{itemize}
    \item On a réussi à créer le pthread dans le main, en testant les codes, on a compris le fonctionnement principal de pthread
    \item Le prochain étape sera implémenter le watchdog qui vérifie un compteur visible uniquement pour la partie Pilotage. Ce compteur sera mis à jour à la fin de la fonction callback dans laquelle on va déménager les commandes de contrôle.
\end{itemize}


\end{document}
